% %
% LAYOUT_E.TEX - Short description of REFMAN.CLS
%                                       99-03-20
%
%  Updated for REFMAN.CLS (LaTeX2e)
%
%\documentclass[a4paper]{refart}
\documentclass[a4paper]{article}
%\usepackage{makeidx}
%\usepackage{ifthen}
\usepackage{lipsum}
% ifthen wird vom Bild von N.Beebe gebraucht!

\def\bs{\char'134 } % backslash in \tt font.
\newcommand{\ie}{i.\,e.,}
\newcommand{\eg}{e.\,g..}
\DeclareRobustCommand\cs[1]{\texttt{\char`\\#1}}

\title{glactools: a suite of utilities for the management of allele frequency information}
\author{Gabriel Renaud \\
gabriel [dot] reno [at]  gmail.com}

\date{}
%\emergencystretch1em  %

%\pagestyle{myfootings}
%\markboth{Changing the layout with \textrm{\LaTeX}}%
%         {Changing the layout with \textrm{\LaTeX}}

\makeindex 

%\setcounter{tocdepth}{2}

\begin{document}

\maketitle

\begin{abstract}
glactools is a set of command-line utilities to store genotype likelihoods and allele count data. It can import data from various formats, merge/filter/split datasets, compute summary statistics and exporting data to various formats.
\end{abstract}

\tableofcontents

\section{Introduction}

glactools aims at storing either :
\begin{itemize}
\item genotype likelihoods for a single individual 
\item allele count information for either a given individual or population.
\end{itemize}
imported from various input sources (VCF, BAM etc), creating intersections, querying the data and exporting it to various formats used by current software tools.




\noindent \begin{tabular}{|l|l|}
\hline
Program & Use \\
\hline
23andme2mistar & Create a mistartools file from 23andme file \\
axt2mistar &  Create a mistartools file from axt files \\
           & representing multiple sequence alignement  \\
bam2mistar &  Create a mistartools file from a BAM file \\
bamtable2mistar & Create a mistartools file from a BAM table file \\
closestMistar & Compute the distance to the nearest \\ 
              & neighboring site for each site \\
epo2mistar & Create a mistartools file from a EPO alignment \\
generateCoords & Generate coordinate (random or windows) \\
               & for a given reference genome \\
mistar2AlleleMatrix & Export a mistartools file to an allele matrix file \\
mistar2bed &  Produce a BED file from the sites in mistartools file \\
mistar2binary & Export a mistartools file to an allele \\
              & matrixtools file of 0=ancestral,1=derived \\
mistar2BinaryPLINK &  Export a mistartools file to the PLINK format \\
mistar2EIGENSTRAT & Export a mistartools file to the EIGENSTRAT format \\
mistar2fasta & Produce a fasta file from a mistartools file \\
mistar2nexus & Produce a nexus file from a mistartools file \\
mistar2segtor & Produce a Segtor file from a mistartools file \\
mistar2treemix & Export a mistartools file to the treemix format \\
mistarcat &  Concatenate multiple mistartools files \\
mistarcompress & Compress a mistartools file to a binary one \\
mistarcompute & Compute summary statistics like \\
              & pairwise differences and D-statistics \\
mistardecompress & Decompress a binary mistartools file \\
mistarfilter & Filter a mistartools file according to certain criterias \\
mistarfreqSpec & Compute the frequency of alleles \\
mistarintersect & Intersect multiple mistartools files \\
mistarmeld & Merge two populations together \\
mistarRenamePop & Change the name of a population \\
mistarstats & How many sites are there in that file \\
mistarunion & Unite multiple mistartools files \\
mistaruniq & Like mistaruniq but returns only the unique lines \\
mpileup2mistar & Convert samtools mpileup to mistartools \\
ms2mistar & Convert ms output to mistartools \\
ms2nj & Create a neighbor-joining tree from a mistartools file \\
vcf2mistar & Create a mistartools file from VCF file \\
vcfcompute & Compute summary statistics directly from VCF files \\
vcfMulti2mistar & Create a mistartools file from VCF file with multiple individuals \\
\hline
\end{tabular}

\section{File format}

\subsection{Text format}

\noindent  A mistartools file is composed of a header and the allele frequency. Each line in the header starts with a \# and has the following format:

\begin{verbatim}
#MISTAR
#PG:[program line]
#GITVERSION: [github revision]
#DATE: YYYY-MM-DD
#chr    coord   REF,ALT root    anc     pop1     pop2     ...
\end{verbatim}

\subsubsection{Genotype likelihoods}

\subsubsection{Allele count}

\noindent The remaining lines have the following format:

\begin{verbatim}
chr    coord   REF,ALT [root info]     [anc info]     [pop1 info]     [pop2 info]     ...
\end{verbatim}

\noindent  The ALT allele is set to N if there were no alternative allele found. The info has the following format:

\begin{verbatim}
[REF allele count],[ALT allele count]:[CPG flag 0=no,1=yes]
\end{verbatim}

\noindent  Here is an example of a line:

\begin{verbatim}
1	1689574	A,N	1,0:0	1,0:0	1,0:0
1	1689575	T,C	0,1:0	1,0:0	1,0:0
\end{verbatim}





\subsection{Indexing}

\noindent If the file was zipped using bgzip, the following command can be used to index it:
\begin{verbatim}
tabix -s 1 -b 2 -e 2 [mistar file]
\end{verbatim}




\section{Importing data}

This program produces a  mistar matrix given a BAM file:

\subsection{From BAM}

\small
\begin{verbatim} 
bam2mistar <options> [name sample] [fasta file] [bam file]  [EPO alignment file]
\end{verbatim} 
\normalsize

\subsection{From VCF}

This program convert VCF files into mistar (prints to the stdout):

\begin{verbatim}
vcf2mistar <options> [vcf file] [name sample] [EPO alignment file]
\end{verbatim}

If it is a vcffile with multiple individuals, use:
\begin{verbatim}
vcfMulti2mistar <options> [vcf file]
\end{verbatim}

\subsection{From 23andme}

This program convert 23andme files into mistar (prints to the stdout)

\begin{verbatim}
23andme2mistar <options> [23andme file] [name sample] [EPO alignment file]
\end{verbatim}

\subsection{From AXT alignment}

This program will parse an axt alignment and print a mistar file

\begin{verbatim}
axt2mistar [chr name] [name sample]  [axt file]
\end{verbatim}


\subsection{From ms simulations}
This program converts ms output into a mistar matrix

\small
\begin{verbatim}
ms2mistar  [mistar file] [correspondance individuals to pop] [size of chromosome]
\end{verbatim}
\normalsize

The correspondence has to have the following format
\begin{verbatim}
pop1:individual1,individual2-npop2:individual3,individual4
\end{verbatim}
The size of the chromsome is the parameter used as -r

\section{Transforming data}
\subsection{cat }
This program concatenates many files where the header is found in the
first file and does not use the headers from the remaining ones
It prints to the /dev/stdout
\begin{verbatim}
mistarcat [mistar file#1] [mistar file#2]
\end{verbatim}


\subsection{filtering}

This program filters a mistartools matrix given certain criterias.

\begin{verbatim}
mistarfilter [mode]
\end{verbatim}

The mode can be one of the following:

\begin{tabular}{ll}
\hline
mode & use \\
\hline
noundef      &   No undefined sites for populations \\
bedfilter    &   Filter mistartools file using sorted bedfile \\
segsite      &   Just retain segregating sites (or trans./transi) \\
popsub       &   Keep a subset of the populations \\
removepop    &   Remove a subset of the populations \\
sharing      &   Retain sites that share alleles between populations \\
nosharing    &   Retain sites that do not share alleles between populations \\
znosharing   &   Retain sites that strickly do not share alleles between populations \\
\end{tabular}

Here is a description of the different filter modes:

\subsubsection{noundef}

This will filter out any site where the allele count is nul (0,0) for both reference and alternative

\begin{verbatim}
mistarfilter noundef [mistar file]
\end{verbatim}


\subsubsection{bedfilter}

This will keep only the positions in the bed file

\begin{verbatim}
mistarfilter noundef [mistar file] [sorted bed file]
\end{verbatim}

\subsubsection{segsite}

This will retain sites where the allele count is greater than 0 for either the reference or alternative for at least one individual. It has options to retain only transitions or transversions. 

\begin{verbatim}
mistarfilter  segsite [options] [mistar file]
\end{verbatim}

\subsubsection{popsub}

This will keep only the population specified in the list. Please note that it will set the alternative allele to 'N' if no population has the alternative allele

\begin{verbatim}
mistarfilter   popsub [mistar file] [comma separated group to keep]
\end{verbatim}

\subsubsection{removepop}

This will remove the population specified in the list.Please note that it will set the alternative allele to 'N' if no population has the alternative allele

\begin{verbatim}
mistarfilter removepop [mistar file] [comma separated group to remove]
\end{verbatim}

\subsubsection{sharing}

This will only retain sites where every individuals in population group 1 share the same allele(s) as every individual in population group 2.
It requires that the allele count for every individual for both groups be non-zero.
A random allele is picked (biased for allele count) for heterozygous position so do not be surprised if you get different outputs every time.

\tiny
\begin{verbatim}
mistarfilter  sharing [mistar file] [comma separated group 1] [comma separated group 2]
\end{verbatim}
\normalsize

\subsubsection{nosharing}


This will filter sites where individuals in population group 1 do not share at least one allele with individual in population group 2.
It requires that the allele count for every individual for both groups be non-zero.
A random allele is picked (biased for allele count) for heterozygous position so do not be surprised if you get different outputs every time.
In other words, the individuals in the first group have to be all reference and the second all alternative or vice-versa.

\tiny
\begin{verbatim}
mistarfilter nosharing [mistar file] [comma separated group 1] [comma separated group 2]
\end{verbatim}
\normalsize

\subsubsection{znosharing}
This will filter sites where individuals in population group 1 strickly do not share any allele with individual in population group 2.
It requires that the allele count for every individual for both groups be non-zero.
Please remember that this will exclude any hetezygous sites

\small
\begin{verbatim}
mistarfilter  [mistar file] [comma separated group 1] [comma separated group 2]
\end{verbatim}
\normalsize




\subsection{intersect}

This program will print the intersection of the mistar files to stdout, it will skip triallelic sites.

\begin{verbatim}
mistarintersect [mistar file 1] [mistar file 2] ...
\end{verbatim}

\subsection{Meld two populations}

This program will merge different specified populations into a single one. You 

\small
\begin{verbatim}
mistarmeld  <options> [mistar file zipped] "popToMerge1,popToMerge2,.." "newid"
\end{verbatim}
\normalsize

Example of usage:
\begin{verbatim}
mistarmeld data.mst.gz "Papuan,Austalian" "oceanians"
\end{verbatim}


\subsection{Rename populations}

This program will rename different specified populations.

\small
\begin{verbatim}
mistarRenamePop <options> [mistar file] "popOldName1,popOldName2,..." "popNewName1,popNewName2,..."
\end{verbatim}
\normalsize

Example of usage:
\begin{verbatim}
mistarRenamePop data.mst "Papuan,Austalian" "Oceanians1,Oceanians2"
\end{verbatim}

\subsection{union}

This program will print the union of the mistar files to stdout, it will skip triallelic sites.

\begin{verbatim}
mistarunion <options> [mistar file 1] [mistar file 2] ...
\end{verbatim}


\subsection{Unique of union}

This program will print the unique union of the mistar files to stdout. Used to merge the same files from different filters.

\begin{verbatim}
mistaruniq [mistar file 1] [mistar file 2] ...
\end{verbatim}


\subsection{replace ancestor}

This program will print the first mistar file but with the ancestral information from the second one to stdout.

\begin{verbatim}
replaceAncestor [mistar file 1] [mistar file 2]
\end{verbatim}


\subsection{use population as root and ancestor}

This program will use specified populations as root and ancestor and produce lines with only those two populations.

\begin{verbatim}
usePopAsRootAnc <options> [mistar file] "poproot" "popanc"
\end{verbatim}



\section{Statistics}

\subsection{Closest sites}

This program will print to stdout the distance to the closest site for each record.

\begin{verbatim}
closestMistar <options> [mistar file]
\end{verbatim}

\subsection{Site frequency spectrum}

This program will print the number of observed alleles for the reference and alternative alleles.

\begin{verbatim}
mistarfreqSpec <options> [mistar file]
\end{verbatim}

\subsection{Summary statistics}

This program takes a mistar matrix and prints some how many lines it has.
\begin{verbatim}
mistarstats  <options> [mistar file]
\end{verbatim}


\subsection{Relationships between individuals}


\subsubsection{Neighbor joining tree}

Compute a neighbor-joining tree using the mistar file.

\begin{verbatim}
mistarcompute nj  <options> [mistar file]
\end{verbatim}

\subsubsection{Pairwise average coalescence}

To compute pairwise average coalescence.
\begin{verbatim}
mistarcompute paircoacompute  <options> [mistar file]
\end{verbatim}

\subsubsection{Pairwise nucleotide differences}

To compute pairwise nucleotide differences.
\begin{verbatim}
mistarcompute pairdiff <options> [mistar file]
\end{verbatim}

\subsubsection{D-statistics}
To compute triple-wise D-statistics
\begin{verbatim}
mistarcompute dstat <options> [mistar file]
\end{verbatim}


\section{Exporting data}
\subsection{To treemix}

To print treemix input.
\begin{verbatim}
mistar2treemix <options> [mistar file]
\end{verbatim}

\subsection{To allele matrix}

This program produces a matrix where each record for population becomes a single allele {A,C,G,T}.
\begin{verbatim}
mistar2AlleleMatrix  [mistar file]
\end{verbatim}


\subsection{To binary allele matrix}

This program takes a mistar matrix and prints the alleles as a binary matrix (0=ancestral,1=derived)

\begin{verbatim}
mistar2binary <options> [mistar file] [comma separated group]
\end{verbatim}

\subsection{To binary PLINK}

This program takes a mistar matrix and prints the genotype and SNP file in PLINK format

\tiny
\begin{verbatim}
mistar2BinaryPLINK <options> [mistar file] [out (.bed)] [out (.bim)] [out SNP file (.fam)]
\end{verbatim}
\normalsize

\subsection{To EIGENSTRAT}
This program takes a mistar matrix and exports the data in EIGENSTRAT.
\tiny
\begin{verbatim}
mistar2EIGENSTRAT <options> [mistar file] [out genotype file (.geno)] [out SNP file (.snp)] [out SNP file (.ind)]
\end{verbatim}
\normalsize

\subsection{To fasta}
This program takes a mistar matrix and exports the data in EIGENSTRAT.
\tiny
\begin{verbatim}
mistar2EIGENSTRAT <options> [mistar file] [out genotype file (.geno)] [out SNP file (.snp)] [out SNP file (.ind)]
\end{verbatim}
\normalsize
\subsection{To fasta}

This program takes a mistar matrix and prints a FASTA file using the allele information with one record per population. Each site generates one base pair.

\begin{verbatim}
mistar2fasta <options> [mistar file] 
\end{verbatim}

\subsection{To NEXUS}

This program takes a mistar matrix and prints the alleles in nexus format.

\begin{verbatim}
mistar2nexus <options> [mistar file]
\end{verbatim}


%\lipsum




\newpage


%%%%%%%%%%%%%%%%%%%%%%%%%%%%%%%%%%%%%%%%%%%%%%%%%%%%%%%%%%%%%%%%%%%%


%%%%%%%%%%%%%%%%%%%%%%%%%%%%%%%%%%%%%%%%%%%%%%%%%%%%%%%%%%%%%%%%%%%%%%

%\input lay_e2

%%%%%%%%%%%%%%%%%%%%%%%%%%%%%%%%%%%%%%%%%%%%%%%%%%%%%%%%%%%%%%%%%%%%%%




\end{document}
